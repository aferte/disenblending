\documentclass[prd,amsmath,aps,floats,amssymb, floatfix, superscriptaddress,nofootinbib,preprintnumbers,twocolumn]{article}  %% removed linenumbers
%\usepackage{lineno} %% added linenumber again (Lucas)
%\linenumbers

\usepackage{makecell}
\usepackage{amsmath,amssymb,natbib,latexsym,times}
\usepackage{graphics}
\usepackage{todonotes,multirow,enumitem}
\usepackage{array}   
\usepackage{breqn}

\title{Estimating the impact of unknown blends on shear and requirements on deblending}
\date{\today}
\author{E. Huff, A. Fert\'e}

\begin{document}

\maketitle



%%%%%%%%%%%%%%%%%%%%%%%%
%%%%%%%%%%%%%%%%%%%%%%%%
\section{Introduction}
\label{sec:intro}

%why big surveys
A major goal of modern cosmology is to understand the origin of the acceleration of the expansion of the universe.
Galaxy surveys allow us to do so, in particular by measuring the growth of large scale structures.
The growth is indeed sensitive to the properties of dark energy or the laws of gravity on cosmological scales, both alternatives to the cosmological constant $\Lambda$ to explain cosmic acceleration (REF).   
The weak gravitational lensing -- the distortion of galaxies images by matter --, the clustering of galaxies and their combination in tomographic bins are great probes of the growth of structures.
Up to now, latest cosmological results from the weak lensing and clustering analysis of DES, HSC and KIDS surveys (REF) showed that the growth is consistent with general relativity predictions and $\Lambda$. 
%systematics in analysis
These analysis require a good handle on systematics effects present in weak lensing data such as the shear calibration or intrinsic alignment.
However the new generation of galaxies surveys -- such as LSST and EUCLID (REF) -- will produce systematics limited measurements, sensitive to the growth of structure.
This forthcoming new gain in data quality will require a new effort in handling (?) systematics in order to obtain the expected level of constraints on cosmology and alternatives to $\Lambda$. 
In this paper, we focus on the impact of blending, a systematics that is subdominant in current surveys such as DES (REF).

%blending
The \textit{blending} of galaxies is due to incorrectly considering two or more galaxies as one (REF). 
This can have different cause: the PSF, the resolution etc (correct???? + REF)
Blending can therefore affect: \\
- photometric redshift estimation, by fitting an incorrect SED model to blended galaxies (correct??);\\
- the distribution of galaxies, by misestimating the number of galaxies;\\
- the shear, by an incorrect estimation of the shear field (TO REFORMULATE). \\
We are here interested in the impact of the blending on the shear, in the context of LSST, EUCLID and WFIRST. 
 
 
 
 

%%%%%%%%%%%%%%%%%%%%%%%%
%%%%%%%%%%%%%%%%%%%%%%%%
\section{Method}
\label{sec:method}

TO DO: we consider 1 z bin
- rederive the modified lensing kernel
- impact on shear: R1, R2 as nuisance parameters.
gamma = R1 g1+ R2/R1 g2
<gamma gamma> = <gamma_single gamma_single > + 2f_b <gamma blend gamma single > + f_b^2 <gamma blend gamma blend>.
vary f_b, R1, R2. 
- Impact of clustering ? (rewrite corr function in term of conditional proba can help understand) 

--> Goal: give requirement on R1, R2 and fb for the deblender. Give prior. 


Note: all of this assume blending of 2 galaxies and that the foreground galaxy of the tblend is not 'opaque' (flux is sum of the two fluxes) 

The ellipticity $e_s$ of a sheared galaxy is: 
\begin{equation}
e_s = e_{s,int} + R \frac{\Sigma (z_{\ell})}{\Sigma_{cr}(z_{\ell},z_s)}
\end{equation}
with $e_{int}$ is the intrinsic ellipticity of the galaxy, R the shear reponse, $z_{\ell}$ is the redshift of the lens galaxy and $z_s$ of the source galaxy.

The ensemble average in the case of a single source is: 
\begin{equation}
<e_s> = \int_0^{z_s} p(z'_s) R \frac{\Sigma (z_{\ell})}{\Sigma_{cr}(z_{\ell},z'_s)} dz'_s.
\end{equation}
We need now to consider the case where the observed image is a blend of two galaxies, where we assume the two galaxies 1 and 2 are independent: 
\begin{multline}
<e_B>  = \int \int dz'_{s_1} dz'_{s_2} p(z_{s_1})p(z_{s_2}) \Sigma(z_{\ell}) \\
\times [H(z'_{s_1}) + H(z'_{s_2})]  \\
\times [ R_1 \int_0^{z'_{s_1}} \frac{dz''_{s_1}}{\Sigma_{cr}(z''_{s_1},z_{\ell})} 
+ R_2 \int_{z'_{s_1}}^{z'_{s_2}} \frac{dz''_{s_2}}{\Sigma_{cr}(z''_{s_2},z_{\ell})}  ]
\label{eq:ellblend}
\end{multline}


The ensemble average for the total ellipticity of a source galaxy is: 
\begin{equation}
<e> = (1 - p_B) <e_s> + p_B <e_B> 
\end{equation}
where $p_B$ is the probability to be a blend. 

The lensing kernel for a galaxy is written as: 
\begin{equation}
g(\chi) = \int d\chi' p(\chi') \frac{f_K(\chi' - \chi)}{f_K(\chi')}
\end{equation}

Assuming equation \ref{eq:ellblend} easily translates to the efficiency (redshift distribution to be dirac function basically gives e), we have: 
\begin{multline}
g = (1 - p_B) g_s + p_B  \int \int d\chi' d\chi'' p'(\chi')p''(\chi'')  \\
\times [ R_1  \frac{f_k (\chi' - \chi)}{f_k (\chi')} 
+ R_2 \frac{f_k (\chi'' - \chi)}{f_k (\chi'')}   ] 
= (1 - p_B) g_s + p_B g_B
\end{multline}
(to rederive)

Now, the convergence power spectrum is written as:
\begin{equation}
P_K^{ij}(\ell) \propto \int_0^{\chi} g_i(\chi') g_j(\chi') P_{\delta} (\frac{\ell}{f_k},\chi') d\chi'
\end{equation}

When we include the case of a blend, we end up with: 
\begin{multline}
P_K^{ij}(\ell) =  (1 - p_B)^2 P_{K}^{i_s j_s}(\ell)  + (1 - p_B) p_B  P_{K}^{i_s j_B}(\ell) \\
+ (1 - p_B) p_B  P_{K}^{i_B j_s}(\ell) + p_B ^2 P_{K}^{i_B j_B}(\ell) 
\end{multline}
where the subscript $s$ stands for a single galaxy and $B$ for blended sources. 

Let's get the expression of e.g. $P_{K}^{i_s j_B}(\ell)$ (similar for $P_{K}^{i_B j_s}(\ell)$) is:
\begin{multline}
P_K^{i_sj_B}(\ell) \propto \int_0^{\chi} g_{i_S}(\chi''') \int_{\chi}^{\infty} \int_{\chi}^{\infty} d\chi' d\chi'' p'_j(\chi')p''_j(\chi'')  \\
[ R_1  \frac{f_k (\chi' - \chi''')}{f_k (\chi')} 
+ R_2 \frac{f_k (\chi'' - \chi''')}{f_k (\chi'')}   ]   P_{\delta} (\frac{\ell}{f_k},\chi''') d\chi'''
\end{multline}

In DES we consider an additive bias in the photometric redshift distributions: $p(z)$ becomes $p(z+b)$, with b the bias, marginalised on. 

Plots of the lensing kernels to see that delta photoz cannot match delta blend. Problem is the integral so try to interpolate the lensing kernel bit and use mesh grid for inregral. 





%%%%%%%%%%%%%%%%%%%%%%%%
%%%%%%%%%%%%%%%%%%%%%%%%
\section{Results}
\label{sec:results}

%%%%%%%%%%%%%%%%%%%%%%%%
%%%%%%%%%%%%%%%%%%%%%%%%
\section{Conclusions}
\label{sec:ccl}






\end{document}

